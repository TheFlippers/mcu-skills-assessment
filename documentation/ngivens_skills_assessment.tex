\documentclass{article}
\usepackage{indentfirst}
\usepackage{blindtext}
\usepackage{graphicx}
\usepackage{wrapfig}
\graphicspath{ {./images/} }

\title{Skills Assessment}
\author{Nathan Givens}
\date{November 8, 2019}

\setlength{\parindent}{4ex}

\begin{document}

  \maketitle

  \section{UART Bus Decoding}

  \subsection{Required Equipment}

  UART bus decoding is a method of capturing communications over a UART bus and
  decoding the information sent between devices. Equipment often decodes the
  information into hexadecimal, but the equipment we have available can display
  the information in binary and ASCII as well. The procedure described in this
  document should apply to any of the oscilloscopes available in the senior
  design labs but was prepared using a Keysight MS0-X 3014T so there may be
  some subtle differences between models.

  \begin{figure}[h]
    \includegraphics[width=\textwidth]{full_scope}
    \caption{An oscilloscope properly decoding data from a UART bus}
  \end{figure}

  \subsection{Making the Measurement}

  Firstly, the oscilloscope must be turned on and the proper number of probes
  must be connected. For UART, two cables are needed to test two way
  asynchronous communications. If only one signal needs to be decoded, then only
  one cable needs to be connected to the oscilloscope. However, having more
  cables connected to the oscilloscope won't hurt anything, it just clutter up
  your area.

  \begin{wrapfigure}{r}{0.4\textwidth}
    \includegraphics[width=30ex]{serial_button_scope}
    \caption{Serial button located between probe configuration buttons 1 and 2.}
  \end{wrapfigure}

  The oscilloscopes have a Serial button located between buttons 1 and 2.
  Pressing this button opens the serial protocol decoding menu at the bottom of
  the display. Here, settings can be adjusted to match expected bus
  configurations.


  \begin{wrapfigure}{r}{44ex}
    \includegraphics[width=44ex]{serial_decode_scope}
    \caption{Serial Decode Menu. Currently shown with UART/RS232 selected.}
  \end{wrapfigure}

  \begin{wrapfigure}{r}{25ex}
    \includegraphics[width=25ex]{mode_menu}
    \caption{Mode selection menu on oscilloscope}
  \end{wrapfigure}

  Inside the menu, selecting the mode button provides a list of communication
  protocols the oscilloscope can decode. This section describes UART decoding,
  so select UART/RS232 from the list. Other sections discuss how to decode some
  of the other protocls listed here. After selcting UART/RS232, the remaining
  entries in the Serial Decode Menu are now specific to decoding UART/RS232.


\end{document}
